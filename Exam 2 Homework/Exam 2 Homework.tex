\documentclass[12pt]{article}  
\usepackage{graphicx, amsmath, amssymb, relsize, parskip}
\usepackage{tikz}
\voffset=-.8cm
\hoffset=-1.5cm
\setlength{\textheight}{24cm}
\setlength{\textwidth}{16cm}
\pagestyle{myheadings}
\newtheorem{q} {Question}
\newcommand{\df}{\displaystyle\frac}
\newcommand{\beq}{\begin{q}\hskip-.2cm)\hskip.2cm}
\newcommand{\eq}{\end{q}\newpage}
\markright {MATH163 Homework 2\hskip2cm Show all your work.}
\begin{document}
{\bf Homework 2 MATH163\hskip1cm Show all your work to get credit. }  \vskip0.2cm 
{\bf Name}: Ayush Pandejee {\bf Due Date}:  \underline{11/04/23} 
\vskip.5cm
%%%questions 
\beq Let $S = \{\emptyset ,a,\{a\}\}$. Determine whether each of these is an element of S, a subset of S, neither, or both.\\

(a) \{a\}\\

The power set of S is P(S) = $\{\emptyset, \{\emptyset\}, \{a\}, \{\{a\}\}, \{\emptyset, a\}, \{\emptyset, \{a\}\}, \{a, \{a\}\}, \{\emptyset ,a,\{a\}\} \hspace{.5mm}\}$\\

\{a\} is both an element of S and a subset of S.\\

(b) \{\{a\}\} \\

Using the Power Set of S from Part A, \{\{a\}\} is a subset of S.\\

(c) $\emptyset$\\

$\emptyset$ is both an element of S and a subset of S. The empty set is a subset of every set, as well as S.\\

(d) $\{\{\emptyset\}, a\}$\\

$\{\{\emptyset\}, a\}$ is neither an element of S nor a subset of S. $\{\emptyset\}$ contains one element, the empty set. It cannot be put with a to form a subset of S nor it is it an alement of S.\\

(e) $\{\emptyset\}$\\

$\{\emptyset\}$ is not an element of S.\\

(f) $\{\emptyset,a\}$\\

$\{\emptyset,a\}$ is a subset of S.\\

\eq
\beq You begin with \$1000. You invest it at 5\% compounded annually, but at the end of each year you withdraw \$100 immediately after the interest is paid.\\

(a) Set up a recurrence relation and initial condition for the amount you have after n years.\\

$a_{n} = 1.05a_{n-1} - 100$, $a_{0} = 1000$\\

(b) How much is left in the account after you have withdrawn \$100 at the end of the third year?\\

$a_{1} = 1.05(1000) - 100$\\

$a_{2} = 1.05(950) - 100 =  897.5 $  \\

$a_{3} = 1.05(897.5)  - 100 = 842.375 $ \\

(c) Find a formula for $a_n$.\\

$a_{n} = 1.05a_{n-1} - 100$ \\

$a_{n-1} = 1.05a_{n-2} - 100$ \\

$a_{n} = 1.05^2(a_{n-2}) - 100(1.05) - 100 $ \\

$a_{n} = 1.05^3(a_{n-3}) - 100(1.05)^2 - 100(1.05) - 100  $ \\

$a_{n} = 1.05^n (a_0) - 100 (\mathlarger \Sigma^{n-1}_{i=0} (1.05)^i )$\\

(d) Use the formula to determine how long it takes before the last withdrawal reduces the balance in the account to \$0.\\

$0 = 1.05^n (a_0) - 100 (\mathlarger \Sigma^{n-1}_{i=0} (1.05)^i )$\\

$0= 1.05^n (1000) - 100(1(\frac{1-1.05^n}{1-1.05})$\\

$0 = 1.05^n (1000) + 2000(1-1.05^n)$ \\

$0 = 1.05^n(1000) - 2000(1.05^n) + 2000$ \\

$0 = -1000(1.05^n) + 2000$ \\

$-2000 = -1000(1.05^n)$ \\

$ 2 = 1.05^n $ \\

$n = 14.2$\\

\eq
\beq  If $P(A)$ means the power set of $A$,\\

(a) Prove that $P(A)\cup P(B) \subset P(A\cup B)$ is true for all sets A and B.\\

Lets assume value $x \in P(A) \cup P(B)$.\\

$x \in (P(A) \vee P(B))$\\

$x \in P(A) \vee x \in P(B) -> x \subset A \vee x \subset B$\\

$x \subset A \vee x \subset B \equiv x \subset (A \vee B)$\\

$x \subset (A \vee B) -> P(A)\cup P(B) \subset P(A\cup  B)$\\

(b) Prove that the converse of (a) is not true. That is, prove that:\\ $P(A \cup B) \subset P(A) \cup P(B)$ is false for some sets A and B.\\

If we set A = $\{1\}$ and B = $\{2, 3\}$\\

P(A) = $\{\emptyset, \{1\} \}$ \\

P(B) = $\{\emptyset, \{2\}, \{3\}, \{2, 3\}  \}$ \\

P($A\cup B$) = $\{\emptyset, \{1\}, \{2\}, \{3\}, \{1, 2\}, \{1, 3\}, \{2, 3\}, \{1, 2, 3\} \}$\\

$P(A) \cup P(B) = \{\emptyset, \{1\}, \{2\}, \{3\}, \{2, 3\} \} $ \\

Let x = $\{1, 3\} $\\

$(\{1, 3\} \in P(A \cup B)) \land (\{1, 3 \} \notin P(A) \cup P(B)$, so the converse of (a) is not true and false for some sets A and B.\\

\eq
\beq Prove that the following is true for all sets A, B, and C: if $A \cap C \subset B \cap C$ and $A \cup C \subset B \cup C$, then $A \subset B$.\\

Assume that $A \notin B$.\\

Case 1: Element $x \in C$:\\

If $x \in C$, then $x \in A \cap C$, and since $A \cap C \subset B \cap C$, we have $x \in B \cap C$. This implies that $x \in B$ because x is in both $B \cap C$ and C.\\

Case 2: Element $x \notin C$:\\

If $x \notin C$, then $x \in A \cap C$, and since $A \cap C \subset B \cap C$, we have $x \in B \cap C$. This implies that $x \in B$ because x is in both $B \cap C$ and C.\\

In both cases, $x \in B$ which contradicts $A \notin B$, concluding that $A \subset B$.

\eq
\beq Let $f:R\rightarrow R$ have the rule $f(x) = \lceil 3x\rceil + 1$and  $g:R\rightarrow R$ have the rule $g(x)=\df{x}{3}$.\\

(a) Find $(f o g)^{-1}(\{2.5\})$.\\

$f o g = \lceil x \rceil + 1 $. However, to get 2.5, $\lceil x \rceil$ has to equal 1.5, which cannot be possible because floor or ceiling functions always up or down to the nearest integer, respectively. So the answer for this is $\emptyset$.\\

(b) Find $(f o g)^{-1}(\{2\})$.\\

$f o g = \lceil x \rceil + 1 $. $(f o g)^{-1}\{ 2\} = x \in (0,1]$ because any number between 0 and 1 will get rounded up to 1 due to the ceiling function.\\

\eq
\beq Find a formula for the recurrence relation $a_n = 2a_{n-1} + 2^n, a_0 = 1$, using a recursive method.\\

$a_n = 2a_{n-1} + 2^n$\\

$a_{n-1} = 2a_{n-2} + 2^{n-1}$ \\

$a_{n-1} = 2(2a_{n-2} + 2^{n-1}) + 2^n = 2^2 (a_{n-2}) + 2^n + 2^n = 2^2a_{n-2} + 2(2^n)$ \\

$a_{n-2} = 2^2(2a_{n-3} + 2^{n -2 }) + 2^n = 2^3(a_n{n-3}) + 2^n + 2(2^n) = 2^3(a_n{n-3}) + 3(2^n) $\\

$a_n = 2^n(a_0) + n(2^n) $ \\

$a_n = 2^n(1 + n) $\\

\eq
\beq Find a formula for an infinite sequence $a_1,a_2,a_3,...$ that begins with the terms 1,2,1,2,1,2,1 and continues this alternating pattern.\\

$a_n = 1 +((n+1) \mod 2)$\\

Basically, with the (n+1) mod 2, any number n that is odd, such as 5, plus 1 mod 2 will equal 0 and any number n that is even, such as 6, plus 1 mod 2 will be equal to 1. Then you just add 1 to get your answer.\\ 

\eq
\beq Find a function $f : {\mathbb Z} \rightarrow {\mathbb N} $ that is one-to-one but not onto.\\\par

$f(x) = 2x - 2$ when $x \geq 0$ and $-2x + 1$ when $x < 0$\\

This works because the function is one to one because no two elements in the domain of f correspond to the same element in the range of f. The function is not onto, however, because the function f's image is not equal to the codomain.\\

Find a function $g: {\mathbb N} \rightarrow {\mathbb Z} $ that is one-to-one and  onto.\\

$g(x) = x/2$ when x is even and $-(n-1)/2$ when x is odd\\

This works because function is one to one because no two elements in the domain of f correspond to the same element in the range of f. The function is also onto because the function f's image is equal to the codomain.\\

\eq
\beq Find  $1+x^2 +x^4 +x^6 +x^8 +... $ assuming  $0<|x|<1$.\\

Sum of geometric sequences: $1/(1-r)$.\\

Since the common difference of the sequence above is $x^2$, the answer is $1/(1-x^2)$.\\

\eq
\beq Use the Principle of Mathematical Induction to  show this inequality is true for all  integers $n\geq 2$:\hskip1cm$\displaystyle\sum_{i=1}^n\df{1}{\sqrt{i\,\,}} > \sqrt{n\,\,}$\\

Base Case (n = 2):

For \(n = 2\), we have:
\[
\sum_{i=1}^2 \frac{1}{\sqrt{i}} = \frac{1}{\sqrt{1}} + \frac{1}{\sqrt{2}} = 1 + \frac{1}{\sqrt{2}}
\]
Now, \(\sqrt{2} < 2\) and \(\frac{1}{\sqrt{2}} > \frac{1}{2}\). Therefore, \(1 + \frac{1}{\sqrt{2}} > 1 + \frac{1}{2} = \frac{3}{2}\), which is greater than \(\sqrt{2}\). So, the base case is true.

Inductive Hypothesis:

Assume that for some positive integer \(k \geq 2\), the inequality
\[
\sum_{i=1}^k \frac{1}{\sqrt{i}} > \sqrt{k}
\]
is true.

Inductive Step:

We need to prove that for \(n = k + 1\), the inequality
\[
\sum_{i=1}^{k+1} \frac{1}{\sqrt{i}} > \sqrt{k+1}
\]
is true.

Starting with the left-hand side:
\[
\begin{aligned}
\sum_{i=1}^{k+1} \frac{1}{\sqrt{i}} &= \left(\sum_{i=1}^k \frac{1}{\sqrt{i}}\right) + \frac{1}{\sqrt{k+1}}
\end{aligned}
\]

By the inductive hypothesis, we know that \(\sum_{i=1}^k \frac{1}{\sqrt{i}} > \sqrt{k}\). Therefore:
\[
\begin{aligned}
\sum_{i=1}^{k+1} \frac{1}{\sqrt{i}} &> \sqrt{k} + \frac{1}{\sqrt{k+1}}
\end{aligned}
\]

Now, we need to prove that \(\sqrt{k} + \frac{1}{\sqrt{k+1}} > \sqrt{k+1}\). We can do this by squaring both sides of the inequality:
\[
\begin{aligned}
\left(\sqrt{k} + \frac{1}{\sqrt{k+1}}\right)^2 &> (\sqrt{k+1})^2
\end{aligned}
\]

Expanding and simplifying the left-hand side:
\[
\begin{aligned}
k + 2\sqrt{k}\cdot\frac{1}{\sqrt{k+1}} + \left(\frac{1}{\sqrt{k+1}}\right)^2 &> k+1
\end{aligned}
\]

Since \(\frac{1}{\sqrt{k+1}} > 0\), we can safely remove it from both sides of the inequality:
\[
\begin{aligned}
k + 2\sqrt{k}\cdot\frac{1}{\sqrt{k+1}} &> k+1
\end{aligned}
\]

Now, subtracting \(k\) from both sides:
\[
\begin{aligned}
2\sqrt{k}\cdot\frac{1}{\sqrt{k+1}} &> 1
\end{aligned}
\]

Simplifying further by multiplying both sides by \(\sqrt{k+1}\):
\[
\begin{aligned}
2\sqrt{k} &> \sqrt{k+1}
\end{aligned}
\]

Squaring both sides again:
\[
\begin{aligned}
4k &> k+1
\end{aligned}
\]

Subtracting \(k\) from both sides:
\[
\begin{aligned}
3k &> 1
\end{aligned}
\]

Since \(k \geq 2\), it's clear that \(3k > 1\), and the inequality is true. Therefore:
\[
\begin{aligned}
\sqrt{k} + \frac{1}{\sqrt{k+1}} &> \sqrt{k+1}
\end{aligned}
\]

Now, we can conclude that:
\[
\begin{aligned}
\sum_{i=1}^{k+1} \frac{1}{\sqrt{i}} &> \sqrt{k} + \frac{1}{\sqrt{k+1}} > \sqrt{k+1}
\end{aligned}
\]

So, the inequality holds for \(n = k+1\).

By the Principle of Mathematical Induction, we have shown that the inequality
\[
\sum_{i=1}^n \frac{1}{\sqrt{i}} > \sqrt{n}
\]
is true for all integers \(n \geq 2\).


\eq
\beq Prove that for all positive integers n,  $3^{2^n} -1$ is divisible by $2^{n+2}$.\\

Base Case: $n = 1$\\

$3^{2^1} - 1 = 3^2 - 1 = 9 - 1 = 8 = 2^{1+2} = 2^3$\\

I.H. Assume that for some positive integer k, $3^{2^k} -1$ is divisible by $2^{k+2}$.\\

I.S. $n = k + 1$\\

\(n = k+1\), \(3^{2^{k+1}} - 1\) is divisible by \(2^{k+3}\).\\

\[3^{2^{k+1}} - 1 = (3^{2^k})^2 - 1\]\\

Difference of squares formula: \(a^2 - b^2 = (a+b)(a-b)\), where \(a = 3^{2^k}\) and \(b = 1\):\\

\[3^{2^{k+1}} - 1 = (3^{2^k} + 1)(3^{2^k} - 1)\]\\

From our inductive hypothesis, \(3^{2^k} - 1\) is divisible by \(2^{k+2}\), so \(3^{2^k} - 1 = 2^{k+2} \cdot m\), where \(m\) is an integer.\\

\[3^{2^{k+1}} - 1 = (3^{2^k} + 1)(2^{k+2} \cdot m)\]\\

Prove that this expression is divisible by $2^{k+3}$, by showing that $3^{2^k} + 1$ is divisible by $(2)$.\\

Since \(3^{2^k} + 1\) is the sum of an even number (\(3^{2^k}\)) and an odd number (\(1\)), it is always an odd number. Any odd number is divisible by 2 once. Therefore, \(3^{2^k} + 1\) is divisible by 2.\\

Hence, \(3^{2^{k+1}} - 1\) is divisible by \(2^{k+3}\).\\

By induction, for all positive integers \(n\), \(3^{2^n} - 1\) is divisible by \(2^{n+2}\).\\

\eq
\beq Find a formula for $$(1-\df1{2^2})(1-\df1{3^2})(1-\df1{4^2})(1-\df1{5^2})...(1-\df1{n^2})$$  where $n \geq 2$, and use the Principle of Mathematical Induction to prove that the formula is correct.\\

Base Case (n = 2):

For \(n = 2\), the formula gives:

\[
\frac{(2^2 - 1)}{2^2} = \frac{3}{4}
\]

The product in the original expression is \((1 - \frac{1}{2^2}) = \frac{3}{4}\). So, the base case is true.

Inductive Hypothesis:

Assume that the formula holds for some positive integer \(k \geq 2\):

\[
(1 - \frac{1}{2^2})(1 - \frac{1}{3^2}) \ldots (1 - \frac{1}{k^2}) = \frac{(2^2 - 1)}{2^2} \cdot \frac{(3^2 - 1)}{3^2} \ldots \frac{(k^2 - 1)}{k^2}
\]

Inductive Step:

We need to show that the formula holds for \(n = k + 1\). Using the inductive hypothesis, we have:

\[
(1 - \frac{1}{2^2})(1 - \frac{1}{3^2}) \ldots (1 - \frac{1}{k^2}) \cdot (1 - \frac{1}{(k+1)^2}) = \left(\frac{(2^2 - 1)}{2^2} \cdot \frac{(3^2 - 1)}{3^2} \ldots \frac{(k^2 - 1)}{k^2}\right) \cdot (1 - \frac{1}{(k+1)^2})
\]

Now, we can simplify the right-hand side using the general form for the formula:

\[
\left(\frac{(2^2 - 1)}{2^2} \cdot \frac{(3^2 - 1)}{3^2} \ldots \frac{(k^2 - 1)}{k^2}\right) \cdot (1 - \frac{1}{(k+1)^2}) = \frac{(2^2 - 1)}{2^2} \cdot \frac{(3^2 - 1)}{3^2} \ldots \frac{(k^2 - 1)}{k^2} \cdot \frac{(k+1)^2 - 1}{(k+1)^2}
\]

We can now expand the product:

\[
\frac{(2^2 - 1)}{2^2} \cdot \frac{(3^2 - 1)}{3^2} \ldots \frac{(k^2 - 1)}{k^2} \cdot \frac{(k+1)^2 - 1}{(k+1)^2} = \frac{(k+1)^2 - 1}{(k+1)^2}
\]

Now, notice that this is exactly the form of the formula for \(n = k + 1\). Therefore, the formula holds for \(n = k + 1\).

By mathematical induction, we have shown that the formula:

\[
    (1 - \frac{1}{2^2})(1 - \frac{1}{3^2}) \ldots (1 - \frac{1}{n^2}) = \frac{(2^2 - 1)}{2^2} \cdot \frac{3^2 - 1}{3^2} \ldots \frac{(n^2 - 1)}{n^2}\]

is correct for all positive integers \(n \geq 2\).

\eq
\beq Which amounts of stamps can be formed using just five cents stamp and nine cents stamp? Prove your answer using strong induction.\\

Base Case:

For small values, we can directly observe that the following amounts can be formed using only 5-cent and 9-cent stamps:

\begin{itemize}
  \item 5 cents: One 5-cent stamp.
  \item 9 cents: One 9-cent stamp.
  \item 10 cents: Two 5-cent stamps.
  \item 14 cents: One 5-cent stamp and one 9-cent stamp.
  \item 15 cents: Three 5-cent stamps.
  \item 18 cents: Two 9-cent stamps.
  \item 19 cents: Two 5-cent stamps and one 9-cent stamp.
  \item 20 cents: Four 5-cent stamps.
  \item 23 cents: One 5-cent stamp and three 9-cent stamps.
  \item 24 cents: Three 5-cent stamps and one 9-cent stamp.
  \item 25 cents: Five 5-cent stamps.
\end{itemize}


Now, we will use strong induction to show that any amount greater than or equal to 32 cents can be formed.

Inductive Hypothesis:

Assume that for all positive integers \(k\) such that \(k \geq 32\), we can form \(k\) cents using only 5-cent and 9-cent stamps.

Inductive Step:

We want to show that for \(n = k + 1\), we can form \((k + 1)\) cents using these stamps.

Consider \(n\) cents where \(n \geq 32\). We have two possibilities:

\begin{enumerate}
  \item Use one 5-cent stamp to form \((n - 5)\) cents. Since \(n - 5 \geq 32\), by the inductive hypothesis, we can form \((n - 5)\) cents using the stamps.

  \item Use one 9-cent stamp to form \((n - 9)\) cents. Since \(n - 9 \geq 32\), by the inductive hypothesis, we can form \((n - 9)\) cents using the stamps.
\end{enumerate}

In either case, we can add one more 5-cent or 9-cent stamp to what was formed for \((n - 5)\) or \((n - 9)\) cents, respectively, to get \((n + 1)\) cents.

Therefore, for any \(n \geq 32\), we can form \((n + 1)\) cents using only 5-cent and 9-cent stamps.

By strong induction, we have shown that any amount greater than or equal to 32 cents can be formed using these stamps.

\eq
\beq  Let P( n) be the statement that a postage of n cents can be formed using just 3- cent stamps and 5- cent stamps. The parts of this exercise outline a strong induction proof that P( n) is true for $n \geq 8$.\\ 

a) Show that the statements P( 8), P( 9), and P( 10) are true, completing the basis step of the proof. \\ 


$3 + 5 = 8, 3 + 3 = 9, 5 + 5 = 10$, making P (8), P (9), and P (10)true.\\

b) What is the inductive hypothesis of the proof?\\ 

I.H.: P(n) is true for $8 \leq n \leq k$, where $k \geq 10$.\\ 

c) What do you need to prove in the inductive step?\\ 

Prove that P(k+1) is true.\\

d) Complete the inductive step for $k \geq 10$ .\\

If $k \geq 10$, then k + 1 = (k - 2) + 3. Since $k - 2 \geq 8$, then P (k - 2) is true due to the I.H.\\

A postage of k - 2 cents can be paid by using 3-cent and 5-cent stamps. Adding one 3-cent stamp, we can pay a postage of k + 1 cents, making P(k + 1) is true.\\

e) Explain why these steps show that this statement is true whenever $n \geq 8$.\\

It shows that for greater than or equal to 8, P(n) is true. When k = 10, and set P(8) as P(k-2), adding a 3-cent coin makes it P(k+1). Getting to the next number, you just have to add k.

\eq 
%%questions
\end{document}
